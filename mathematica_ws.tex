\documentclass{exam}

\usepackage[utf8]{inputenc}
\usepackage{amsmath}
\usepackage{amsfonts}
\usepackage{amssymb}
\usepackage{listings}

\newcommand{\Z}{\mathbb{Z}}
\newcommand{\N}{\mathbb{N}}
\newcommand{\Q}{\mathbb{Q}}
\newcommand{\R}{\mathbb{R}}
\newcommand{\C}{\mathbb{C}}
\newcommand{\F}{\mathbb{F}}

\lstset{language=Mathematica}

\title{UWA MU Mathematica WS Examples}
\date{April 7,2022}
\author{MU - Dathan Tran}
\begin{document}
\maketitle
\begin{questions}

    \question Determine the limit $\displaystyle\lim_{x\to0} \frac{a^x-1}{x}$ using \lstinline[columns=fixed]{Limit[]} (use \lstinline[columns=fixed]{f1} to see how to use this function!) 
    \question Define $f(x,y)=\sin(x)\tan(y)-x^2\sqrt{y}$. Use Mathematica to determine $f(1,2),f(1,3),f(0.5,5)$
    \question Consider the curves $C_1$ and $C_2$ defined by $\mathbf{r}_1(s)=(1-s,s,3+s^2), s\in \R$ and $\mathbf{r}_2(t)=(t-2,3-t,t^2), t\in \R$ respectively.

    \begin{parts} 
        \part Construct $\mathbf{r}_1(s)$ and $\mathbf{r}_2(t)$ as vectors in Mathematica
        \part Solve for $P$, the single intersection of the two curves, and the values of $s$ and $t$ using \lstinline[columns=fixed]{Solve[]} (you may want to look up the documentation of \lstinline[columns=fixed]{Solve[]} using \lstinline[columns=fixed]{f1})
        \part Determine the angle between the tangent vectors at point $P$ using Mathematica. 
    \end{parts}

    \question
        Consider the matrix $A=\left(\begin{array}[2]{cc} 1 & 4 \\ 3 & 2\end{array}\right)$. 
        \begin{parts}
            \part Determine $A^{30}$ and use \lstinline[columns=fixed]{MatrixForm[]} to view the matrix
            \part Determine $A^{-1}$ using \lstinline[columns=fixed]{Inverse[]}
            \part Use \lstinline[columns=fixed]{Eigensystem[]} to determine the eigenvalues and eigenvectors of $A$
        \end{parts}
    \question Consider $f(x_y)=y^2+x^2(1+2y)$. Use Mathematica to determine $f_{xx},f_{yy}$ and $f_{xy}$.
    \question Let $f_1(x,y)=3x^9+5y^2, f_2(x,y)=6x^7+2y^8$. Determine the Jacobian Matrix of $f=\left(\begin{array}[1]{c} f_1(x,y) \\ f_2(x,y)\end{array}\right)$. It may be useful to define $f$ as 
    \begin{lstlisting}
        f = {f1,f2}
    \end{lstlisting}
    And using \lstinline[columns=fixed]{Grad[]} on \lstinline[columns=fixed]{f}.

    \question Evaluate the integral $\displaystyle\int_1^a \dfrac{e^{-ax}}{\sqrt{a}x}\mathrm{d}x$ for $a=2,3,4$ to 2SF.

    \question Plot $f(x)=\sin(x^2)\,x^2$ for $x\in(-3,3)$ and rescale the plot vertically to be between $(0,8)$ using \lstinline[columns=fixed]{PlotRange}

    \question Plot $x(x-a)^3$ for $a=1,2,3$ on the same plot and make their curves blue, green and red respectively using \lstinline[columns=fixed]{PlotStyle}.

    \question Plot the vector field $f(x,y)=(\sin(x)\cos(y)+x^2,x+y)$ using \lstinline[columns=fixed]{StreamPlot[]} and \lstinline[columns=fixed]{VectorPlot[]}.

    \question Plot the surface $z=\dfrac{\sin(\sqrt{x^2+y^2+c^2})}{\sqrt{x^2+y^2+c^2}}$ for
        \begin{parts}
            \part c=1 using \lstinline[columns=fixed]{Plot3D[]}
            \part $c\in (1,10)$ using \lstinline[columns=fixed]{Manipulate[]} or \lstinline[columns=fixed]{Animate[]} to visualise how $c$ changes the function
        \end{parts}
\end{questions}
\end{document}
